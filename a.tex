Tak, format, w którym zapisuje się wzory matematyczne takie jak \( f(x) = -(x^{-1}) \), to LaTeX. LaTeX jest szeroko stosowanym systemem do składu tekstów naukowych, zwłaszcza tych zawierających dużo matematyki. 

LaTeX pozwala na wygodne i eleganckie formatowanie skomplikowanych równań, tablic, grafik i innych elementów tekstu. Oto kilka prostych przykładów, jak wyglądają podstawowe elementy matematyczne w LaTeX-u:

1. Proste równość:
   \[E = mc^2\]
   Kod: `\[ E = mc^2 \]`

2. Ułamek:
   \[\frac{a}{b}\]
   Kod: `\[ \frac{a}{b} \]`

3. Potęga i pierwiastek:
   \[
   x^2 \quad \text{oraz} \quad \sqrt{x}
   \]
   Kod: `\[ x^2 \quad \text{oraz} \quad \sqrt{x} \]`

4. Pochodne:
   \[
   \frac{d}{dx} f(x) = f'(x)
   \]
   Kod: `\[ \frac{d}{dx} f(x) = f'(x) \]`

5. Całki:
   \[
   \int_a^b f(x) \, dx
   \]
   Kod: `\[ \int_a^b f(x) \, dx \]`

LaTeX jest bardzo potężny i oferuje wiele zaawansowanych możliwości formatowania tekstu i matematyki. Jest szeroko stosowany w środowiskach akademickich, szczególnie w dziedzinach matematyki, fizyki, informatyki i inżynierii. Aby korzystać z LaTeX-a, można używać różnorodnych edytorów LaTeX, takich jak Overleaf, TeXShop, MiKTeX, TeXstudio, i wiele innych.